\documentstyle[color,subfigure,graphicx,epsf,here,cite,otf,comment,nccmath,mediabb,fancyhdr,12pt]{jarticle}

\graphicspath{{./pic/}}
%%%\documentstyle{jarticle}
\setlength{\textwidth}{16.2cm}%A4
\setlength{\textheight}{23cm}%A4
\setlength{\topmargin}{-1.5cm}
\setlength{\oddsidemargin}{0cm}
\setlength{\evensidemargin}{0cm}
\setlength{\parskip}{1pt}
%\pagestyle{myheadings}
\pagestyle{fancy}
\lhead[名前]{近藤徹多}
\rhead[\today]{\today}
\title{課題2 医療画像認識}
\author{近藤徹多}
\date{\today}

\begin{document}

	\maketitle
	\vspace*{20pt}

	\begin{center}
		{\LARGE \bf 基本課題}
	\end{center}
    %\setcounter{section}{0}
    \addtocounter{section}{-1}
	\section{level1}
        時系列長が30で固定された1次元データの波形をDTWを用いて識別する.
        具体的には2つ時系列データの時刻をt1, t2とすると,t1, t2全ての組の距離を計算し,
        それらの合計が最小になるような距離(経路)を求める.
        距離は誤差の二乗でを計算する.
        また, testデータのクラスはreferenceデータとtestデータの距離が最小になるreferenceデータのクラスとして識別する.
        識別結果を表\ref{tb:result1}に示す.

        \begin{table}[]
            \centering
            \caption{時系列長30の1次元データの識別結果}
            \begin{tabular}{|c|c|}
                \hline
                test data & class \\ \hline
                test1     & 1     \\ \hline
                test2     & 1     \\ \hline
                test3     & 1     \\ \hline
                test4     & 2     \\ \hline
                test5     & 2     \\ \hline
                test6     & 2     \\ \hline
            \end{tabular}
            \label{tb:result1}
        \end{table}
            


	\section{level2}
	    時系列長が30で固定された3次元データの波形をDTWを用いて識別する.
        3次元データなので, 距離はベクトルとして誤差の二乗を計算する.
        識別結果を表\ref{tb:result2}に示す.

        \begin{table}[]
            \centering
            \caption{時系列長30の3次元データの識別結果}
            \begin{tabular}{|c|c|}
                \hline
                test data & class \\ \hline
                test1     & 1     \\ \hline
                test2     & 2     \\ \hline
                test3     & 1     \\ \hline
                test4     & 2     \\ \hline
                test5     & 2     \\ \hline
                test6     & 1     \\ \hline
            \end{tabular}
            \label{tb:result2}
        \end{table}
	
	\section{level3}
	    level2のデータをランダムに間引きしたり延長した3次元データの波形をDTWを用いて識別する.
        DTWは長さの異なる時系列データにも対応できるので, level2と同様に計算できる.
        識別結果を表\ref{tb:result3}に示す.


        \begin{table}[]
            \centering
            \caption{ランダムな時系列長の3次元データの識別結果}
            \begin{tabular}{|c|c|}
                \hline
                test data & class \\ \hline
                test1     & 1     \\ \hline
                test2     & 2     \\ \hline
                test3     & 2     \\ \hline
                test4     & 1     \\ \hline
                test5     & 1     \\ \hline
                test6     & 1     \\ \hline
            \end{tabular}
            \label{tb:result3}
        \end{table}
		
	\section{level4}
	    時系列長が256で固定された64次元データの波形をDTWを用いて識別する.
        今回もベクトルとして距離計算を行う.
        また, referenceデータが各クラス3つある.
        初めに, referenceデータを各クラス1つずつ用いて識別し,
        次にreferenceデータを全て用いて, 平均距離を比較して識別を行う.
        referenceデータを各クラス1つずつ用いた識別結果を表\ref{tb:result4-1},
        referenceデータを全て用いた識別結果を表\ref{tb:result4-2}に示す.

        \begin{table}[]
            \centering
            \caption{時系列長256の64次元データの識別結果(referenceデータ1つずつ)}
            \begin{tabular}{|c|c|}
                \hline
                test data & class \\ \hline
                test1      & 2     \\ \hline
                test2      & 2     \\ \hline
                test3      & 2     \\ \hline
                test4      & 1     \\ \hline
                test5      & 1     \\ \hline
                test6      & 1     \\ \hline
                test7      & 2     \\ \hline
                test8      & 2     \\ \hline
                test9      & 2     \\ \hline
                test10     & 2     \\ \hline
                test11     & 2     \\ \hline
                test12     & 2     \\ \hline
                test13     & 1     \\ \hline
                test14     & 2     \\ \hline
            \end{tabular}
            \label{tb:result4-1}
        \end{table}

        \begin{table}[]
            \centering
            \caption{時系列長256の64次元データの識別結果(referenceデータ全て)}
            \begin{tabular}{|c|c|}
                \hline
                test data & class \\ \hline
                test1      & 1     \\ \hline
                test2      & 1     \\ \hline
                test3      & 1     \\ \hline
                test4      & 1     \\ \hline
                test5      & 1     \\ \hline
                test6      & 1     \\ \hline
                test7      & 1     \\ \hline
                test8      & 2     \\ \hline
                test9      & 2     \\ \hline
                test10     & 2     \\ \hline
                test11     & 2     \\ \hline
                test12     & 2     \\ \hline
                test13     & 1     \\ \hline
                test14     & 2     \\ \hline
            \end{tabular}
            \label{tb:result4-2}
        \end{table}

\end{document}
